\documentclass{beamer}
\usepackage[spanish]{babel}
\usepackage{tikz} 
\usepackage{tikz-network}
\usepackage[utf8]{inputenc}
\usepackage{algorithm,algorithmic}
\usetheme{CambridgeUS}
\usepackage{url}
\usepackage{natbib} 
\title{Tarea 1: Divulgación Científica}
\author{Lic. Arnoldo Del Toro Peña }
\institute[UANL]{Universidad Autónoma de Nuevo León}
\titlegraphic{ \includegraphics[width=2cm]{R.jpg} \hfill \includegraphics[width=4cm]{fime.png} }
\newcommand{\M}{M_0, M_1, \cdots }
\newcommand{\X}{X_0, X_1, \cdots, X_n}
\newcommand{\Y}{Y_0, Y_1, \cdots, Y_n}
\newtheorem{definicion}{Definición}
\newcommand{\divul}{Divulgación Científica}
\newcommand{\difu}{Difución Científica}

\begin{document}
	
	\begin{frame}
		\titlepage
	\end{frame}
	
	\begin{frame}{Tabla de contenidos}
		\tableofcontents
	\end{frame}
	
	\section{Divulgación Científica}
	
	\begin{frame}{Definiciones}
		\begin{definicion}[\divul]<1->
			Se le llama divulgación científica al conjunto de actividades que interpretan y hacen accesible el conocimiento científico al público general, es decir, a todas aquellas labores que llevan el conocimiento científico a las personas interesadas en entenderlo o informarse de él  \citep{fundora_divulgacion_2021}.
		\end{definicion}
	
		\begin{definicion}[\difu]<2->
			La difusión es la propagación del conocimiento entre especialistas y constituye un tipo de discurso diferente, contiene un conjunto de elementos o signos propios de un discurso especializado y una estructura que se constituyen en factores clave a la hora de su evaluación \citep{espinosa_santos_difusion_2010}.
		\end{definicion}
	
	\end{frame} 
	
	\begin{frame}{Bibliografía}
		 \setcitestyle{notesep={; },round,aysep={,},yysep={;}}
		\bibliography{biblio}
		\bibliographystyle{kluwer}
	\end{frame}

\end{document}
